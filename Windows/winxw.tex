\documentstyle[a4]{article}
\begin{document}
\title{XXWord v2.1\\The Crossword Solver's Apprentice}
\author{by Graham Wheeler\\(c) 1994--1998  All Rights Reserved}
\maketitle
\tableofcontents

\section{Welcome}

Welcome to XWord!

XWord is a crossword solver's helper. It uses some efficient
dictionary searching algorithms for computing the set of possible 
solutions to a partially complete crossword. It can then show you 
which possible letters can be placed in each square, and which 
possible words can go through each square. The idea is that you use 
XWord as you solve the crossword, entering letters as you determine
them, and using the feedback provided by XWord as additional clues.
XWord can also sometimes surprise you by completing a half to
two-thirds finished crossword correctly (although this depends on
giving XWord some additional information).

XWord is distributed as part of WordsWorth. As such, it uses the
same dictionary as WordsWorth, which does not contain proper nouns.
This restricts its usefulness somewhat, but the problem may be
addresses soon (Pentiums help!).

DISCLAIMER - This software is distributed without a warranty of 
any sort. You use this software at entirely your own risk. I will
not be responsible for any loss, damage, injury, death, disaster,
plagues, famine, divorces, disagreements, etc that may result from
the use of this software.

\section{Setting up a Crossword Grid}

XWord is quite a powerful tool with a number of features. The
best way to learn how to use it is to get a crossword and start
using it. This manual gives a brief description of the process
that you should follow to use XWord to help you solve the crossword.

The first task is to set up a grid. Choose `New' from the
File Menu. Enter the size of the grid in the dialog box (you can
use the sliders to change the values). The grid will then be shown 
with all squares set to black. You now have to enter the layout of
the white square. In the top left-hand corner is a red rectangle -
this is the board cursor position. You can move it around
the board with the cursor keys, or by clicking the mouse on a
part of the grid. To mark a square as empty (white), put the
cursor on it and type '?' (space will also do). To reset it 
to black, put the cursor on it and enter '#'. To enter a known 
letter at a square, put the cursor on it and type the letter.

As you enter square contents, the cursor moves automatically,
generally following the direction you establish initially
by pressing the right or down cursor keys. An exception is 
when entering a letter, where the movement follows the direction
of empty white squares.

It is a good idea to set up the board with just `#' and `?' before
entering any letters. Boards generally have some symmetry - this
is exploited by having some commands to flip and reflect the 
grid in various ways. These are available from the `Grid' menu.
You can also set the default symmetry to use `on the fly' as you
enter the grid, from the `Options' menu. As most crosswords are
reflected and inverted about the center, it is a good idea to use
this as the default symmetry (called `reflect and flip' in the
menu).

After entering the grid, it is a good to idea to prevent accidental
changes by `locking' it (selected from the `Grid' menu). You should
then enter any letters that you already know. Once you have done
that, you need to give XWord additional information about the clues
that you still have to solve. This is described in the next section.


\section{Giving XWord Additional Clue Information}

XWord uses its dictionary to work out all possible solutions to the
crossword. However, if there is anything wrong with the board layout
constraints, the results will be unpredictable, and may include
XWord concluding that the crossword has no solution at all.

There are two ways these constraints can incorrect: you may have
entered letters on the board that are wrong, or XWord is being led
astray by the limitations of its dictionary. There is not much that 
can be done about the former, but there is some hope in the latter
case!

Firstly, you need to mark any multi-word clues as being multiple words.
You can do this using the mouse, as is described in the next section,
or using the keyboard, which we describe here.

Move on to the letters that start new words (other than the first
word) and use one (or both) of the following keys:

\begin{description}
\item [^] Mark square as start of a new down word (brown);
\item [>] Mark square as start of a new across word (magenta)
\end{description}

The square will be displayed with a red line at the left edge
and/or the top edge, indicating the start of the new word.
It is a good idea to also mark squares which follow hyphens,
as WordsWorth's dictionary does not include any hyphenated words.

If you suspect that there are words that are not in the dictionary
(such as proper nouns), you need to tell XWord to not consult
the dictionary for that word. Move on to each letter of the word,
and mark it with one of the following:

\begin{description}
\item [|] Don't limit current square to admissible down words
\item [-] Don't limit current square to admissible across words
\item [+] Don't limit current square by dictionary at all (a combination
of the above two)
\end{description}

These attributes will appear as horizontal and/or vertical 
lines through the center of the square.

XWord will automatically add such override attributes to any word
in the grid that is longer than the longest word in the XWord
dictionary (13 letters at the time of writing).

\section{Using the Mouse to define the Grid}

To set up the grid with the mouse, you can click on a square
to select it (the cursor will move to the square). If the square
is already selected, then clicking with the {\em right} mouse button
will bring up a dialog box which will allow you to set the options
for the square. 

You can also
specify an allowed subset of letters that can be used for the
square. This is useful in the case when you know that a clue 
is an anagram of a certain set of letters, but you don't know 
which anagram (you would presumably have done a dictionary 
consultation to find the anagram, so this is most likely to
occur when the consult failed to find the anagram, usualy
because it is a proper noun). You can then at least tell XWord
that it should only allow the known letters to be used for each
square in that word (unfortunately you need to do this for
each square in the word; on the other hand this gives a finer
level of control). Actually, the new Anagram attribute mentioned 
below makes this feature largely redundant, but it has been left 
in in case it proves useful.

Clicking on the selected square with with the {\em left} mouse button brings
up a dialog box that lets you specify settings for a whole clue
(down and/or across). You can specify the length of the word(s)
down and/or across, and whether the dictionary should be overridden
down and/or across. You can also specify multi-word clues by
separating the word lengths with commas. For example, a clue whose
answer consists of a two letter word followed by a three letter
word should be entered as 2,3.

Two additional attributes that can be set are the `concept keys'
for the word, and, if you know the word is an anagram of some 
other word, you can enter that other word here too. Using these
attributes slows down the computation of constraints by XWord
quite considerably, but they can be extremely useful. For more
information about concept keys, see Section~\ref{conckey}.

You can use any combination of mouse and keyboard actions when
setting up the board. XWord will handle these consistently.
The fastest way to set up a board is to use the left mouse button
to enter all the clues which have squares in the top left quadrant
(if the grid has two axes of symmetry) or in the top half of the
grid (if the board has one axis of symmetry), save the result,
use the Grid menu to complete the other three-quarters or half
of the grid, and then enter any dictionary overrides and multi-word
clue information in the rest of the grid, either with the keyboard 
or the mouse.

\section{Solving the Crossword}

After entering the additional info, select {\tt Enable Constraints}
from the {\tt Solve} menu and then select {\tt Compute Contraints} from that
menu (or click on the white area surrounding the grid; this has the
same effect). XWord will attempt to solve the crossword, or
at least find the possible words that can be put in the grid, based
on its dictionary, and the constraints you have entered on the grid.
If you haven't filled in many letters yet, this may take a few minutes.

You can then move the cursor around to squares for which you 
do not know the letter (and which XWord didn't fill in when
you asked it to solve the crossword).  As you move about, you
will see two things changing. At the bottom of the screen XWord
displays all characters that it believes can be entered in the
current square. If the square is empty, on the right hand side
of the screen you will see up to 8 down words and 8 across
words that can go through the square. The total number of possible
words across and down is also shown.

As you move around, XWord works out all possible
words that can go down and across through that square, and hence
what letters can be placed on the square. When the board is nearly
empty, the number of possible words shown on the right may be in
the thousands. When it is a lot less than that, but still more than
eight, you can view the entire list by selecting {\tt View Across} or
{\tt View Down} from the {\tt Solve} menu.

You now proceed by using the lists of possible words and letters that
XWord displays, together with the crossword clues, to work out
more letters. Every time you enter another letter XWord can
narrow down the possibilities further. If XWord finds that a
square can have only one possible letter, it will display the letter 
in red, without actually committing the letter to the board permanently.
Solving the crossword then becomes a collaborative process between 
you and XWord. 

If the number of possible words is ever shown as zero, then you have
a problem. Somewhere you have made an error in setting up the grid,
or (often the case) there is a word that is not in the dictionary
and for which you have not told XWord to override the
dictionary (or you have specified concept key restrictions that are
not matching the right word, or incorrect anagram restrictions).
As a result, XWord has concluded that the crossword has no solution. 
You will have to try to identify the problem and correct it (this can
be tricky - a good approach is to move around the grid, discarding the
constraints (from the `Solve' menu) before each move, and look for
places on the grid that still show zero possibilities - these are
the potential problem areas).

HOWEVER, THERE IS ONE CASE IN WHICH SEEING ZERO POSSIBILITIES IS OK!
This is if the word in the crossword
is longer than the longest word in the dictionary. If this is the case,
XWord will automatically override the resulting artificial constraint 
in any perpendicular cross words. You will notice this happening often
in the demo version, where the dictionary is limited to words of six
letters or less; any words of more than six letters will show up with
zero choices, but the words that cross it will be unaffected.

\section{The Menus}

\subsection{The File Menu}

The file menu in Xword mode lets you save and load your crosswords
and grids, start new grids, print the grid to a printer, or quit XWord. 

\subsection{The Solver Menu}

\begin{description}
\item [Enable Constraints] - Enabling constraints turns on XWord's
word matching as you move about the grid. Usually you will not do
this until you have finished entering the grid, as it serves no
purpose at that stage. When constraints are enabled, moving around
the grid with the keyboard is much slower.
\item [Disable Constraints] - If you have enabled constraints but 
find that you need to move around the grid quickly with the keyboard
for some reason, you can use this to disable constraint computation.
\item [Compute Constraints] - XWord will work out the current 
constraints on the entire grid. The more empty squares on the grid,
the longer this will take. Nontheless, it is usually acceptable to
do this each time you solve a clue and enter it in the grid. A
fast way to do this is to click on the white area surrounding the
grid.
\item [Discard Constraints] - XWord discards the contraints it
has worked out. XWord will do this automatically if you change 
the grid in any way other than entering a letter, but just in case
you can use this menu option to force a discard.
\item [View Across] this pops up a window with a list of all the
words that can pass through the current square in a horizontal
direction.
\item [View Down] this pops up a window with a list of all the
words that can pass through the current square in a vertical
direction.
\end{description}

\subsection{The Tool Menu}

\begin{description}
\item [Consult Dictionary] this is identical to dictionary
consultations in WordsWorth. It is discussed further below.
\item [Dump Dictionary] allows you to dump the words in the current
dictionary to a plain text file.
\item [Build Dictionary] allows you to convert a plain text file
containing alphabetically sorted words, one per line, into the
compressed format used by WordsWorth and XWord.
\item [Interrupt Build] allows you to interrupt a dictionary build;
The build can be resumed later. The reason for allowing the build
to be interrupted and resumed is that it can take several hours.
\item [Resume Build] allows you to resume an interrupted dictionary build.
\item [Register] allows you to register the program so that all
features and dictionaries can be used.
\end{description}

\subsection{The Grid Menu}

\begin{description}
\item [Flip Grid] Mirror-reflect top half of board setup
left-to-right;
\item [Reflect Horizontally] Copy left half board setup to right
half;
\item [Reflect Vertically] Copy top half board setup to bottom half.
\item [Set Grid Symmetry] This lets you set the `on-the-fly'
symmetry to use while the grid is being entered. Initially
the default symmetry (set from the `Options' menu) is used;
you can use this option to change to a symmetry other than
the default;
\item [Lock/Unlock Grid] Locking the grid prevents any further
alteration of the basic black square/white square layout. 
This is useful once you have entered the grid, to prevent it
from being accidentally modified, which can happen quite 
easily if you enter a word with automoving on, and the 
current automove direction is not set correctly. Unlocking
the grid will allow you to edit it again.
\item [Enable/Disable Automove] This lets you turn cursor
automoving on or off.
\end{description}

\subsection{The Options Menu}

\begin{description}
\item [Configure Colours] This lets you choose the colours 
that XWord should use when drawing the grid;
\item [Configure Options] This lets you set the default
crossword size and symmetry, and select the dictionary
file that should be used.
\end{description}

\input{consult.tex}

\input{concept.tex}

The obvious place to use concept keys with XWord is when you have clues
consisting of a single word. You can then enter that word as a concept
key. It is sometimes a good idea to try a dictionary consultation with
the concept key first - you may find that there are only a few matches,
none of which is the word that you are looking for, in which case using
the concept key in the grid would be a bad idea.

\section{Crossword Save File Format}

For those who are interested in such things, this section describes the
format of saved crossword files used by XWord.

The file begins with the dimensions of the board, and the basic grid layout.
Here is an example:

\begin{verbatim}
13 13

CHAFF#S.....E
A#.#OMAHA#.#R
N....#GENUINE
A#.#..O..#.#C
SANTA##TEAPOT
TA......#.#.#
AR..#.#.#...P
#O#.#...H...E
.N..K.##A...R
.#.#O.D.R#.#V
....A.O#ACCRA
.#.#L.V.R#.#D
RAMPAGE#ELOPE
\end{verbatim}

Because we only have a
single letter to represent each square, if a square contains
a letter and a flag or more than one flag, not all of the information
can be represented. However, in this case the board can be followed by
additional lines of the form:

\begin{verbatim}
<row> <column> <`command' character>
\end{verbatim}

or:

\begin{verbatim}
<row> <column> <`command character'><argument>
\end{verbatim}

to specify this additional information. The full set of 
command characters that can be used is:

\begin{verbatim}
       	.       Empty square
       	?       Empty square
       	#       Black square
       	^       Start of new down word in multi-word clue
       	>       Start of new across word in multi-word clue
       	+       Do not restrict this square at all
       	|       Do not restrict this square in a downward direction
       	-       Do not restrict this square in an across direction
       	A-Z     Put this letter in the square
       	=       Concept keys across
        :       Concept keys down
       	&       Anagram across
        *       Anagram down
\end{verbatim}

The last four are followed by a string argument containing the concept
key list or anagram letters.

Here is an example of such a file as saved by XWord:
      
\begin{verbatim}
      15 15
      
      ; Created by XWord

      #......#......#
      #.#.#.#.#.#.#.#
      ....#..........
      #.#.#.#.#.#.#.#
      ......#........
      #.###.#.#.#.#.#
      .........#....#
      ###.###.###.###
      #....#.........
      #.#.#.#.#.###.#
      ........#......
      #.#.#.#.#.#.#.#
      ..........#....
      #.#.#.#.#.#.#.#
      #......#......#
      
      
      2 11 >
      6 7 ^
      8 10 >
      8 12 >
      8 13 |
      9 13 |
      10 13 ^
      11 7 ^
      12 6 >
\end{verbatim}
      
Note that saved board files can have comment lines starting with 
semicolons. However, these CANNOT appear before the board dimensions!

\end{document}
