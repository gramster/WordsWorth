% todo - board editing, saved game formats, clean up
% check whether dictionary overrides are allowed when 
% cheating is disabled.

\documentstyle[a4]{article}
\begin{document}
\title{WordsWorth v2.1\\The `Ultimate' CrossWord Game}
\author{by Graham Wheeler\\(c) 1992--1998  All Rights Reserved}
\maketitle
\tableofcontents

\section{Introduction}

\subsection{Welcome}

Welcome to WordsWorth!

WordsWorth was inspired by an article in Communications of the ACM in
May 1988, entitled `The World's Fastest Scrabble Program'.  I had
always been a frustrated Scrabble player. I take ages to move,
generally play pretty badly anyway, and dislike waiting for my
opponent. As a result, I decided to write my own program, which would
have both infinite patience and lightning response time {\tt ;-)}. 

It soon became clear that the principles involved were not bound by
the rules of Scrabble as such, but more by the concept of crosswords.
Imposing rules simply implied restricting the set of possible moves.
The idea of a general-purpose crossword game playing engine was born!

Please send your comments, bug reports, flames, and anything else to
me at {\tt gram@oms.co.za}.  I have tried to make WordsWorth
as generic and general as possible; if you have suggestions,
particularly about further generalisation, let me know! I am also
interested in hearing about erroneous words or omissions in the
dictionary.

If you have word lists in other languages that you would like made
into WordsWorth dictionaries, send them to me and I will do it for
you (e-mail only please). The dictionary builder runs only under
UNIX, takes many hours to run, and requires about 16Mb of RAM, so it
is not distributed with WordsWorth.

Prior to v2.x, WordsWorth was MS-DOS-based. WordsWorth v2.0 was the
first version to run under both MS-Windows and UNIX. MS-DOS is no longer
supported. The dual platform nature of WordsWorth v2.x has been made
possible through the use of wxWindows, a cross-platform C++
application framework.

A few features of the MS-DOS version are not available in v2.x. These
are:

\begin{itemize}
\item the hint facility is no longer present. The same effect can be
achieved with a combination of `auto', `next', and `undo'.
\item the ability to exclude words from the dictionary has been 
removed. The dictionary now has far fewer bad words than in versions 
earlier than v1.3, and the performance penalty of using an exclusion
dictionary is no longer worth paying. If WordsWorth plays an invalid
word, use the `next' button to force it to play a different
move.
\item blank tile exchanges are no longer supported.
\end{itemize}

Some new features were added:

\begin{itemize}
\item the program is now almost entirely mouse driven. Only in
crossword solving mode is keyboard input necessary.
\item WordsWorth now supports from one to four players; previous
versions were fixed at two.
\item more level control options have been introduced.
\item you can now view the contents of the tile pool at any stage.
\item all of the configureable options can now be set from within
the program.
\item a new option allows you to explicitly choose which tiles you
want to draw at the end of a move (computer players still use random
drawing only).
\end{itemize}

DISCLAIMER - This software is distributed without a warranty of 
any sort. You use this software at entirely your own risk. I will
not be responsible for any loss, damage, injury, death, disaster,
plagues, famine, divorces, disagreements, etc that may result from
the use of this software.

\subsection{Variations on a Scrabble Theme}

WordsWorth allows you to design your own crossword games.  You can
create any size and shape board up to a 21{\tt x}21 square, and
pepper it with black crossword squares that can't be played upon.
You can make any square a single, double, triple, quadruple, etc
letter or word score type. You can play free-form on the board, or be
forced to build on to the existing structure (as in Scrabble). You
can play a territorial game in which you have to build on to your own
structure. You can specify what letters are available, how much they
are worth, and give them a strategic `weight'. You can save and
restore games, and undo and redo all the moves.  If you like
Scrabble, you can use WordsWorth as a Scrabble opponent (although the
dictionary is not exactly the same as the OSW).

\subsection{WordsWorth Players}

WordsWorth supports from one to four human or computer players, and
provides several options for assisting in multi-player human games.
You can control the level and style of your computer opponent(s).  A
future version of WordsWorth will allow multi-player games over the
Internet.

\section{Game Overview}

The rules of the game are quite straightforward. They are described
in their loosest form here, as they vary somewhat depending on the
configuration. The scoring method is explained in stages in the
following sections.

\subsection{The Objective}

The basic aim is to make words in a crossword-like fashion from a
set of letter tiles which you draw out of a common pool. You 
alternate plays with the program, and when the pool is exhausted 
and one player has no more tiles, or both players are stuck, the
game ends. You score according to the word you make, as well as 
other new words that may result from your play. For example, if
the board contains:

\begin{verbatim}
    DOG
\end{verbatim}

and you play the word {\tt STOP} thus:

\begin{verbatim}
    DOGS
       T
       O
       P
\end{verbatim}

you get score for both {\tt DOGS} and {\tt STOP}. On the other hand, you 
{\em must} make valid words with your play, so

\begin{verbatim}
        S
    DOGST
        O
        P
\end{verbatim}

is not a valid way to play the word {\tt STOP}, as {\tt DOGST} is invalid.

Hyphenated words and proper nouns (capitalised words) are not
allowed. The rules of Scrabble now admit the American `-ize' as well
as the English `-ise' spelling of words, and WordsWorth follows this
policy.

At the end of the game, the scores are compared to see who wins.
The scores are calculated according to the location of the word
(that is, what special squares did it cover), and the letters
played, each different letter having a different score value.
If either or both players still hold unplayed tiles, the score 
of those tiles is deducted to reach a final score.

Usually, you must build on to the existing words on the board
(the exception being the first move, of course). WordsWorth
offers some flexibility in these matters, however.

\subsection{The Board}

Play takes place on a board which can be any size that will fit
within a $21\times21$ square area. If you want a rectangular board you
specify the dimensions; if you want some other shape board you 
specify the size of a rectangle that can contain it, and use
black squares to create the shape you desire.

\subsection{Hot Squares}

Apart from black squares, the scoring of plays is affected by the
type of the squares on which the play takes place. Thus, a square 
may be a `double letter square', `triple letter square',
and so on, and similarly for `double word', etc. We will call such
squares `hot squares'. If your move
results in one or more hot squares being covered,
you take this into account when working out the score for the move.
Hot squares that affect letter scores are taken into account before those
that affect the whole word.

Thus, if you play a two-letter word in which one letter has score
1, and the other score 4, and the first letter is played on a
double word square while the second is played on a triple letter
square, then the score for that word is $2\times(3\times4+1) = 26$. Note
that this is just for that word; the score for that move will
usually involve score for one or more other `cross words' as well. Once
a hot square is covered with a tile, its type ceases to have
any effect on scores; thus, only when you actually place a tile on
a hot square does it have an effect on the score (once again,
WordsWorth allows this to be changed).

\subsection{The Tile Pool}

Each player draws tiles out of a common pool. Usually this is 
fixed in size, and each player draws a fixed number of tiles
randomly from the pool at the start of each game, replenishing
their tile rack at the end of each move. In Scrabble, for example,
each player holds seven tiles at a time.

WordsWorth also allows you to play any tile from the pool without
drawing a small set of tiles each move. The pool can also be
unlimited, allowing you to play any letter you like at
any time, until all players get stuck anyway.

\subsection{Making a Move}

The basic idea behind making moves was described earlier. There 
are also some other types of moves. For example, if you cannot
make a good word, you can pass your turn. If you do so, you have
the option of returning any or all of your current tiles to the
pool, and drawing replacements (assuming you are drawing tiles,
and not playing freely from the pool as described in the previous 
section).

\subsection{Blank Tiles}

There is a special tile that plays a similar role to a joker in
cards. This tile is blank, and can be used to represent any
character. If you have such a tile and play it, you must state
what character it is representing. When such a tile is on the
board the letter it represents is shown in red; other tiles are 
displayed using black text.
If you are passing your turn and wish to discard a blank (a rare
event!), you must enter it as an underscore `{\tt \_}', as it does
not represent any particular character (other than itself) when
being discarded. 

%WordsWorth optionally allows either player to pick up a blank 
%tile from the board if the letter it symbolises is placed in its
%place by that player. Thus, if you played `{\tt SToP}', and the program
%held an `{\tt O}' tile, it could exchange its `{\tt O}' for the blank. This
%may constitute a full move on its own, a part of a pass move, or
%simply something that can be done at the start of any move; you
%specify whether you want these moves to be allowed at all, and 
%if so, when.

\subsection{The First Move}

The first move may optionally have some constraints placed on how it
should be played. For example, it could be specified that the word
must be played in the third row, or that it must cross the center
square, or that it may be played anywhere at all. When using a square
board, these constraints would usually stipulate that the first word
may be played in either direction and must pass through the center
square on the board (assuming the board has an odd-numbered size).
Alternative constrainst are most useful when a board is being used
which is not square, or which has some unorthodox layout built using
black squares.

\subsection{Subsequent Moves}

Usually each move other than the first move must build on to the
existing words like a crossword. Other possibilities are territorial
play (where each player builds on their own words), and free-form
play, in which play may occur anywhere. In the former case,
attachment to the opponents words is allowed, but every play must
include a word with a previously- placed letter that was placed by
the same player.

A more precise description, useful in understanding some of
WordsWorth's level-control options, is that every word played must
include at least one `anchor square'. Exactly what constitutes an
anchor square depends on the type of play. For normal, Scrabble-like
play, any square which contains a played tile is an anchor square.
For territorial play, any square that contains a tile played by the
current player is an anchor square. For freeform play, all board
squares are anchor squares.

\section{Playing the Game}

This section gives a brief overview of how to use WordsWorth,
intended to get you up and playing as soon as possible.

\subsection{Starting a Game}

To start playing a game, you must select `New Game' from the `File'
menu. A dialog box will pop up, asking you to choose the type of
game, the number of players, and whether cheating is allowed. When
you are happy with the selections, you should click on the `OK'
button.

The board will then be drawn, with the first player's tile rack being
shown below. The tiles may have their faces exposed, or may all
appear blank; the latter is intended to ensure privacy when there 
are multiple human players; clicking on any the tiles will expose
them to view.

\subsection{What you see on the Screen}

To the right of the board is a panel containing several buttons. For
now, we will concern ourselves with only three of these: `Pass',
`Done', and `Run'.

At the botton of the WordsWorth window is a status line. At this
stage it should display a message saying either `Player 1 get ready'
(if the tiles are hidden), or `Player 1 make your move' (if they
are exposed).

\subsection{Making WordsWorth Play First}

If you want the role of player 1 to be adopted by the
program, you should click on the `Run' button. WordsWorth will
make the first move, and all subsequent moves for the first player.
On the other hand, if you click on the `Auto' button, WordsWorth
will autoplay a single move.

\subsection{Playing the First Move Yourself}

To play the move yourself, you need to click on each tile in the
rack that you want to play, followed by the location on the board
where you want it to be played. As you click on a rack tile, it
will turn red to show that it is selected; when you click on the 
board location it will turn grey, and a copy (also grey) will be
placed on the board. You can also select a tile from the rack by
pressing the corresponding key.

If you click on the wrong location on the board, just click on the
right one, and the letter will be moved. To undo a letter placement,
click on the grey letter on the board (if you are playing from 
a rack, as opposed to playing from the tile pool or an unlimited
pool, you can also click on the grey letter in the rack to undo;
in the other cases clicking on the rack letter will instead select
a second copy of the tile to play).
With a little bit of experimentation you'll quickly get the hang
of it.

Once you have placed all the tiles, you should click on the `Done'
button to tell WordsWorth the move is done. The move will be checked,
and if acceptable the next player will be prompted to move. If there
is a problem an error message will appear on the status line and you
will have to try again.

If the word that you play (or a new cross word) is not in the
dictionary, you will be given the option of overriding the 
dictionary and forcing WordsWorth to accept the move.

\subsection{Passing}

If you want to pass instead of playing a word, click on the `Pass'
button. A message will appear on the status line asking you to select
what tiles you wish to discard. Click on the tiles, and then on
`Done'.

If you want to save your game you can do so by selecting `Save Game'
from the File menu; use `Load Game' to restore a saved game.

\section{The Game Buttons}

The full set of panel buttons and actions are:

\begin{itemize}
\item {\em Pass}, which tells WordsWorth that you want to pass this
round. You will be given a chance to discard tiles; click on the
`Done' button to complete the move.
\item {\em Done}, which tells WordsWorth that you have finished your
move. The move will be checked for validity, and if it is admissible,
the next player will be given a chance to play. If there is a problem
with the word, a description will be shown on the status line, and
you will  have to try again.
\item {\em Auto}, which tells WordsWorth to play the current move.
\item {\em Next}, which tells WordsWorth to play the current move,
or, if a `Next' or `Auto' has already been done, to undo the last
move and play it differently (WordsWorth stores up to twenty possible
moves which it chooses from. If you keep clicking on `Next' it will
eventually cycle back to the original move).
\item {\em Run}, which tells WordsWorth to make all moves for the
current player. 
\item {\em Stop}, which tells WordsWorth to stop autoplaying moves
for any players for which you have clicked on `Run'.
\item {\em Undo}, which undoes the last move. You can undo multiple
moves all the way to the start of the game.
\item {\em Redo}, which tells WordsWorth to redo the last undone
move. You can redo multiple moves. Note that if you undo one or more
moves, and then play a different move, you will not be able to redo
the undone moves.
\end{itemize}

If you specified at the start of the game that cheating was not allowed
then the `Auto', `Next' and `Undo' buttons will have no effect during
that game.

\section{Invalid Moves}

If you enter an invalid move, WordsWorth will not accept it when
you clock on `Done'. Instead, an error message will appear in the
status area. The possible errors are:

\begin{description}
\item [1st word must cross row $<${\em n}$>$]
You are playing the first word on the board and you have to play
it according to constraints which you have violated. These constraints
are specified in the configuration.
\item [1st word must cross col $<${\em n}$>$]
You are playing the first word on the board and you have to play
it according to constraints which you have violated. These constraints
are specified in the configuration.
\item [Word has no anchor square!]
The word does not attach to the existing word structure on the
board. You must build on to what is already there (but see below).
\item [Word conflicts with board!]
You entered a word which contains a letter which conflicts with
a letter already on the board.
\item [You don't have a blank!]
The word you entered requires that you have a blank tile, which
you don't.
\item [You don't have a $<${\em char}$>$]
The word you entered requires that you have a $<${\em char}$>$
tile, which you don't.
\item [Bad Xword]
The word you entered does not form all valid cross words.
\item [Can't start there!]
The word you entered must have its first letter preceded by an
empty square or the edge of the board, not another letter.
\item [Can't end there!]
The word you entered must have its last letter followed by an
empty square or the edge of the board, not another letter.
\end{description}

If one of these errors occurs, you will simply have to play a
different move. You will be given the option of overriding
WordsWorth's dictionary lookups of the main word and any cross words,
in case you wish to play a move that is not in the dictionary.

\section{Board Edit Mode}

WordsWorth is always in one of two `modes': game mode, 
or edit mode. Game mode (the default) is used for playing games, 
while edit mode is used for creating new board types for games.

The menu bar at the top is different for each mode, as is the panel
on the right. In game mode, the panel contains the play buttons,
the round and move number, the number of tiles left in the pool, and
the player scores, while in edit mode the panel is empty.

To enter edit mode, click on `Board Layouts' in the Options menu 
in game mode.

You will then be presented with a dialog box which allows you to 
specify which board you are editing, its size, and any constraints
on the first move. When you have set the values appropriately and
click on `OK', the edit board will appear. You can then click on 
any square on the board; a dialog box will appear which allows you to
set the attributes for that board square.
To return to game mode, select `Save and Exit' or `Exit without
Saving' (whichever is appropriate) from the File menu.

\section{The File Menu}

When in game mode, the File menu has the following options:

\begin{description}
\item [New Game] allows you to start a new game. If a game is
already being played, you will be asked whether you want to discard
it. You will then be presented with a dialog box which allows you to
choose the game type, the number of players, and whether or not
cheating is allowed (if you disallow cheating, several WordsWorth
options will be disabled; this is intended for multi-player games
where there are two or more human players participating).
\item [Load Game] allows you to restore a saved game and
continue playing.
\item [Save Game] allows you to save the current game being
played to disk.
\item [Print] prints the current board to a printer.
\item [Print Setup] lets you control some printer options.
\item [Print Preview] lets you preview and print the current board 
to a printer.
\item [Quit] exits WordsWorth.
\end{description}

\section{The Options Menu}

The Options menu allows you to design new game types, and change some options 
that affect game play.

In WordsWorth, a `game type' consists of:

\begin{itemize}
\item a board type;
\item a tile pool type;
\item a set of rules.
\end{itemize}

Board types and tile pool types are defined separately, so that they
can be reused (for example, we may wish to create several different 
game types which all use the same board and pool types, and differ
only in the rules).

A board type has a name, a size (in rows and columns),
a layout, and a set of restrictions on allowed opening
moves.

A tile pool type has a name, and the number and score of each of the
27 different tiles.

From the Options menu, you can create new game types, new board types,
and new pool types. You can also editing existing games, boards and
pools. In addition, you can change how tiles are displayed on the
screen, how WordsWorth will choose automoves for each of the (up to)
four players, change some settings affecting whether tiles are
exposed or not, and so on. The options are:

\begin{description}
\item [Misc] pops up a dialog box allowing
you to set the default number of players (used to initialise
the dialog that pops up when you select `New Game'), and the path
of the dictionary file. You can also enable or disable three options:

\begin{description}
\item [Hide racks between moves] If this option is selected, then
players will have to click on their racks to see the tiles. This is
for when there are multiple human players, and ensures that when a
player completes their move they don't get to see the next player's
tiles (at least not without the other player knowing that they've
seen them!).
\item [Show auto-player racks] If this option is enabled, you will be
able to see the contents of the rack when a computer opponent is
making a move.
\item [Use different colours for players] This option draws each
players tiles in a different colour. It is automatically enabled in
territorial play games.
\end{description}

Note that changes made to these settings take effect immediately.
On the other hand, changes to the configuration of a game, board, or
pool, will only have an effect from when the next game is started
(so you can't change the layout of the board while a game is in
progress).

\item [Board] lets you edit an exiting board type, or
create a new board type. You will be presented with a dialog box
which allows you to choose which board type you want to edit. 
The dialog box also allows you to set or change the name of the
board type, the size of the board, and the constraints on the first 
move. 

To create a new board type, you should click on the `Board' pull-down
list box and select the `Create New' board type.

Once you have set the values in the dialog box to your satisfaction,
click on the OK button. WordsWorth will switch into Edit mode.

\item [Pool Size] lets you create a new pool type, or edit
the name and/or letter counts of existing pool types. A dialog box will
appear which has 27 sliders for setting the tile counts. 
You can choose which pool you want to edit from the list box at the
top left of the dialog; select `Create New' to create a new pool.

Note that you must click on `Save Settings' for the results to be
saved. Clicking on `Exit' exits the dialog box but does not save any
changes that have been made.
\item [Letter Scores] is almost the same as the Pool Size
option, except it allows you to set the scores of the letter tiles,
rather than the number of each tile.
\item [Letter Weights]
\item [Level] lets you change the settings WordsWorth uses to choose 
automoves for each player.
\item [Game] allows you to create new game types or edit
existing game types. Once again you can choose the game type to edit
from a list box, and select `Create New' to create a new game type.
The options that you can set for a particular game type are:

\begin{itemize}
\item the name of the game;
\item the name of the board type to use;
\item the name of the pool type to use (may be unnecessary);
\item how new words must be played - whether the game is
crossword-like, territorial, or free-form;
\item whether the player is playing from a rack, directly from the
pool, or whether any letter can be played at any stage (in the last
case you need not specify a pool type name above);
\item the number of tiles in the rack; this is only relevant if
playing from a tile rack;
\item the number of tiles that must be played to get a score bonus,
and the number of bonus points to award when that many tiles are
played.
\item whether the player can choose which tiles they draw after
playing a word; if not selected tiles are drawn randomly from the
pool. Again, this option only has an effect if tiles are to be played
from a rack, as in the other two cases players do not draw tiles.
\end{itemize}
\end{description}

\section{The Tools Menu}

The Tools menu has the following commands:

\begin{description}
\item [Show Pool] pops up a window showing the current set of
tiles that are left in the tile pool.
\item [Consult] allows you to perform dictionary consultations.
\item [Dump Dictionary] allows you to dump the words in the current
dictionary to a plain text file.
\item [Build Dictionary] allows you to convert a plain text file
containing alphabetically sorted words, one per line, into the
compressed format used by WordsWorth and XWord.
\item [Interrupt Build] allows you to interrupt a dictionary build;
The build can be resumed later. The reason for allowing the build
to be interrupted and resumed is that it can take several hours.
\item [Resume Build] allows you to resume an interrupted dictionary build.
\item [Register] allows you to register the program so that all
features and dictionaries can be used.
\end{description}

\section{Autoplay Level Options}

You can set the autoplay levels for each player from the `Level' item
in the `Edit' menu. A dialog box will appear with a number of
options:

\begin{description}
\item [Player] Use this listbox to select which player's settings you
wish to change;
\item [Use adaptive strategy] If you select this option, the player's
strategy will change depending on how well it is doing with respect to
the other players. 
\item [Avoid opening up hot squares] If you select this option, the
autoplayer will avoid making moves which make it easy for the next 
player to score double/triple/etc word scores. This does tend to make
the playing style more cramped, but this can be compensated for by
increasing the weighting on word length.
\item [Use letter weights] If this option is selected, then the
letter weights will be used as well as the score in choosing a move
(if it is not selected, letter weights are only used when an 
autoplayer passes a move and discards tiles).
This can be used, for example, to make an autoplayer tend to play
some letters as soon as possible (such as a Q, for example), while
tending to hold on to other letters (such as blanks and S's). The
rationale is both to do with penalties for unplayed tiles at the
end of the game, and the idea that blanks and S's can often be best
used in high-scoring plays (where the score will compensate for the
weight).

If this all sounds like greek to you, don't worry - XWord by
default comes with useable weight settings. You can turn this option
on if you like, just to see the effect, but it isn't too important
except for those interested in experimenting with different
strategies.

Note that this option is sometimes (but not always) used in Adaptive
Play.

\item [Minimum allowed score] This lets you set a lower limit on
the acceptable score for an autoplayer's moves. Be warned that if
you set this too high the player will tend to pass more often than
play a word.
\item [Maximum allowed score] This lets you set a ceiling on the
score that an autoplayer can get with a move. It gives human players
an unfair advantage, but if you're not a great player you may 
appreciate being able to do this.
\item [Minimum score to play blank] This option is similar to the
letter weights in intent. It is used to cause an autoplayer to
hold back on using a blank tile until it can use it in a high-scoring
play.
\item [Minimum allowed length] This option lets you specify the
minimum allowed length of autoplayer moves.
\item [Maximum allowed length] This option lets you specify the
maximum allowed length of autoplayer moves.
\item [Word length weight] This option lets you specify how
much importance an autoplayer will attach to the length of the
word it plays. Essentially this is used to modify the score by
an amount equal to the product of the value you sepcify and the
length of the word. For example, if you set this to two, and 
WordsWorth finds a two-letter move with score 10, and a five-letter
word with score 6, it will choose the five letter word as it
considers it to have a score of 6+2*5=16 while the other word
it considers to score 10+2*2=14.
\item [Maximum new cross words] WordsWorth has an infuriating
habit of playing several adjacent words in the same direction,
forming lots of two letter and then three letter cross words
(it can do this as it has so many two and three letter words
in its dictionary). This can make it quite hard for a human 
player at times. This option lets you limit the number of cross
words that it can make per move. Setting this to one or two 
makes the autoplayer play a more `open' game. I find this a
very useful option.

\item [Dictionary filter factor] This is a real technical one
to explain. Suffice it to say that it allows you to force WordsWorth
to only look at some of the words in the dictionary. The higher
you set it, the less words it will check. The actual choice of
words is random, but will tend to favour shorter words. While
not quite accurate, you can think of the value as being the
percentage of words that are skipped.

Generally I don't advise the use of this option, unless you are
playing a free-form game. In that case XWord can take a long
time to play as there are so many possible moves. Setting the 
dictionary filter  factor can be used to cause significant 
speed ups (but of course the autoplayer will not play that well).
\item [Anchor filter factor] This is another option to speed up
play, at the cost of weakening the autoplayer. When used moderately,
it is less drastic than the dictionary filter factor.

The concept of an anchor square was described earlier. The anchor
filter factor is the percentage of anchors that should not be
considered for play. Again, in free-form play this can cause
significant speedups. Setting it to ninety should result in a
ten times speedup; setting it to ninety-five should cause a 
twenty times speedup, and so on (put another way, setting it
to ninety-five means only five percent, or one-twentieth, of the
anchor squares will be considered).

I don't recommend setting this to much higher than fifty, except 
in the case of free-form play.
\end{description}

Setting any of the sliders to zero causes that option to have no
effect.

After changing the settings for a player, click on `Save Player'.
When you have finished making all your changes, click on `Exit'.
Note that clicking on `Exit' does not itself result in any changes
being saved; you must click on `Save Settings' for that.

\input{consult.tex}

\input{concept.tex}

\end{document}
