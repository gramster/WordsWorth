\documentstyle[a4,12pt]{article}
\begin{document}
\title{WordsWorth v1.2\\The `Ultimate' CrossWord Game}
\author{Graham Wheeler\\(c) 1992,1993,1994  All Rights Reserved}
\maketitle
\tableofcontents

\section{Welcome}

Welcome to WordsWorth!

WordsWorth was inspired by an article in Communications of the 
ACM in May 1988, entitled `The World's Fastest Scrabble Program'.
I had always been a frustrated Scrabble player. I take ages to 
move, generally play pretty badly anyway, and dislike waiting
for my opponent. As a result, I decided to write my own program,
which would have both infinite patience and lightning response
time {\tt ;-)}. 

It soon became clear that the principles involved were not bound
by the rules of Scrabble as such, but more by the concept of crosswords.
Imposing rules simply implied restricting the set of possible moves.
The idea of a general-purpose crossword game playing engine was born!

It has taken a long time to complete (I had a lot of other things to
do {\tt 8-(} ), but I think it has been worth it. WordsWorth allows you
to design your own crossword games. If you like Scrabble, you
can use WordsWorth as a Scrabble opponent (although not with the
official Scrabble dictionary); or you can be creative, and design 
your own games.

You can create any size and shape board up to a 21{\tt x}21 square, and
pepper it with black crossword squares that can't be played upon.
You can make any square a single, double, triple, quadruple, etc
letter or word score type, up to 11 times! You can play free-form
on the board, or be forced to build on to the existing structure
(as in Scrabble). You can play a territorial game in which you have
to build on to your own structure. You can specify what letters are
available, how much they are worth, and give them a strategic `weight'.
You can save and restore games, and undo and redo all the moves.
And you can control the level and style of your computer opponent.
The possibilities are immense. Now it is up to you!

Please send your comments, bug reports, flames, and anything
else to me at {\tt gram@gramix.aztec.co.za}.
I have tried to make WordsWorth
as generic and general as possible; if you have suggestions,
particularly about further generalisation, let me know! I am
also interested in hearing about erroneous words or omissions
in the dictionary.

If you have word lists in other languages that you would like
made into WordsWorth dictionaries, send them to me and I will
do it for you (e-mail only please).

DISCLAIMER - This software is distributed without a warranty of 
any sort. You use this software at entirely your own risk. I will
not be responsible for any loss, damage, injury, death, disaster,
plagues, famine, divorces, disagreements, etc that may result from
the use of this software.

The last public release of WordsWorth before this was v1.1d.
That and this version offer:

\begin{itemize}
\item improved {\tt -S}/{\tt -s}/{\tt -Q} handling;
\item bugs fixed in reporting bad cross words, score handling,
game loading, and tile drawing;
\item the {\tt buildict.exe} utility and the exclusion dictionary,
plus the new command `{\tt E}';
\item a single cleaned up dictionary of over 80,000 words;
\item adds the {\tt XWORD.EXE} utility (this is like the Consult command, but
also does multi-word anagrams).
\end{itemize}

\section{Configuration}

If you are new to WordsWorth, you should skip this section,
and run the included {\tt SETUP.EXE} program to create a
configuration file. The {\tt SETUP} program will ask you a number
of questions about, amongst other things, your video hardware,
what board layout you require, and so on, and then write a 
suitable configuration file to disk. You can then run
WordsWorth ({\tt WW.EXE}) directly. When you want to explore the 
possibilities of WordsWorth further, or maintain multiple 
configurations, you should read this section.

WordsWorth is highly configurable. At this stage you need
to be able to use a text editor to change the configuration,
as it is stored in plain text files. WordsWorth includes
several example `configuration definition files' which can
be combined together into configuration files using the 
supplied {\tt SETUP} program. However, you will probably want to 
learn how to create your own configuration definitions at 
some stage.

The configuration elements have names which are mostly
self-explanatory. A comment begins with a`{\tt \#}' character;
everything from that character to the end of the line is 
ignored. Whitespace in the file is also ignored; this is
why the mask tiles, draw tiles, and score have to be
separated by special field delimeter characters in the
saved game files. Things that can be changed include the
size and layout of the board, bonus scores, number,
distribution and score of tiles, and so on. The various
configuration entries are described in Appendix A of the 
{\tt README} file in the WordsWorth directory (this file is an 
electronic version of this manual with additional information). The
{\tt SETUP} program itself can also be reconfigured, as it is
driven by entries in a file named {\tt SETUP.CFG}. The layout
of this file is described in Appendix C of the {\tt README} file.
You can modify the {\tt SETUP.CFG} file to suit your own requirements.

Normally, WordsWorth reads the configuration from the file 
{\tt WW.CFG}, which is the file created by the {\tt SETUP} program. You
can override this choice on the command line, and create
different sets of configurations in different files. By using
configuration definition files, the size of the top-level 
configuration files can be kept small, as common sets of
entries can be `factored out'.

The order of entries in the configuration is not important,
except that the number of rows and columns in the board must
be specified before the constraints on the first move and the
specification of the board layout.

You can include configuration definition files in a top-level 
configuration file by using lines of the form:

\begin{verbatim}
        Include = myfile
\end{verbatim}

where `{\tt myfile}' is the name of the configuration definition file
to include. WordsWorth will immediately read the specified file,
and then resume reading the top level file. Note that only one
level of nesting is allowed; `{\tt Include}'d files cannot themselves
include other files. 

The unregistered version of WordsWorth is provided with only 
a subset of the dictionary, containing about 15000 short words.
The full dictionary contains over 80,000 words. In addition, 
the registered version of WordsWorth includes a utility to
specify a list of words that are to be excluded from the
dictionary. The `{\tt BUILDICT.EXE}' program reads a {\em sorted} list
of words all in {\em upper case}, and produces a dictionary from 
them. To exclude the words in this list from play, the dictionary
file should be named `{\tt WORDS.NO}'. For example, if the word list
is in the file `{\tt WORDS.EX}', you can do this with:

\begin{verbatim}
        BUILDICT -O WORDS.NO WORDS.EX
\end{verbatim}

There is a batch file, `{\tt REBUILD.BAT}', that will sort the words
in `{\tt WORDS.EX}' and then run `{\tt BUILDICT}' on them.
See section~\ref{exclude} for more about excluding words from play.

The ability to exclude words from the dictionary that WordsWorth
can play allows you to overcome imperfections that you may 
discover in the dictionary.

WordsWorth is shipped with the following conguration definition 
files:

\begin{tabular}{|l|l|}
\hline
Name & Description\\
\hline
{\tt herc.def}	&	Definitions for Herc/Mono display adaptor\\
{\tt ega.def}	&	Definitions for EGA color display adaptor\\
{\tt vga.def}	&	Definitions for VGA color display adaptor\\
{\tt svga.def}	&	Definitions for Super VGA display adaptor\\
 & \\
{\tt color.def}	&	Color/texture definitions for color systems\\
{\tt mono.def}	&	Color/texture definitions for mono systems\\
 & \\
{\tt diamond.def} &	Diamond shaped board\\
{\tt grid.def}	  &	Grid shaped board\\
{\tt scrabbrd.def}&	Scrabble board\\
{\tt board21.def} &	$21\times21$ square board\\
{\tt board7.def}  &	$7\times7$ baby square board\\
 & \\
{\tt alice.def}	&	Letter amounts/scores based on `Alice in
Wonderland'\\
{\tt cia.def}	&	Letter amounts/scores based on `CIA Fact
File'\\
{\tt bible.def} &	Letter amounts/scores based on King James
Bible\\
{\tt jargon.def}&	Letter amounts/scores based on the Jargon
File\\
{\tt scrabtil.def} &	Letter amounts/scores based on Scrabble\\
 & \\
{\tt level.def}	&	Supplementary definitions for game rules\\
\hline
\end{tabular}

{\em Make backups of your configuration files before changing them!}

\section{Game Overview}

The rules of the game are quite straightforward. They are described
in their loosest form here, as they vary somewhat depending on the
configuration. The scoring method is explained in stages in the
following sections.

\subsection{The Objective}

The basic aim is to make words in a crossword-like fashion from a
set of letter tiles which you draw out of a common pool. You 
alternate plays with the program, and when the pool is exhausted 
and one player has no more tiles, or both players are stuck, the
game ends. You score according to the word you make, as well as 
other new words that may result from your play. For example, if
the board contains:

\begin{verbatim}
    DOG
\end{verbatim}

and you play the word {\tt STOP} thus:

\begin{verbatim}
    DOGS
       T
       O
       P
\end{verbatim}

you get score for both {\tt DOGS} and {\tt STOP}. On the other hand, you 
{\em must} make valid words with your play, so

\begin{verbatim}
        S
    DOGST
        O
        P
\end{verbatim}

is not a valid way to play the word {\tt STOP}, as {\tt DOGST} is invalid.

Hyphenated words and proper nouns (capitalised words) are not
allowed.

At the end of the game, the scores are compared to see who wins.
The scores are calculated according to the location of the word
(that is, what special squares did it cover), and the letters
played, each different letter having a different score value.
If either or both players still hold unplayed tiles, the score 
of those tiles is deducted to reach a final score.

Usually, you must build on to the existing words on the board
(the exception being the first move, of course). WordsWorth
 offers a lot of flexibility in these matters, however.


\subsection{The Board}

Play takes place on a board which can be any size that will fit
within a $21\times21$ square area. If you want a rectangular board you
specify the dimensions; if you want some other shape board you 
specify the size of a rectangle that can contain it, and use
black squares to create the shape you desire. See for example 
the `{\tt grid.def}' and `{\tt diamond.def}' configuration definition files.

Apart from black squares, the scoring of plays is affected by the
type of the squares on which the play takes place. Thus, a square 
may be labelled as a `double letter', `triple letter', `4 $\times$ letter',
and so on, and similarly for `double word', etc. If your move
 results in one or more of these types of squares being covered,
you take this into account when working out the score for the move.
Squares that affect letter scores are taken into account first. 
Thus, if you play a two-letter word in which one letter has score
1, and the other score 4, and the first letter is played on a
double word square while the second is played on a triple letter
square, then the score for that word is $2\times(3\times4+1) = 26$. Note
that this is just for that word; the score for that move will
usually involve score for one or more other words as well. Once
a square is covered with a tile, however, its type ceases to have
any effect on scores; thus, only when you actually place a tile on
such a square does its type have an effect on the score.

\subsection{The Tile Pool}

Each player draws tiles out of a common pool. Usually this is 
fixed in size, and each player draws a fixed number of tiles
randomly from the pool at the start of each game, replenishing
their tile rack at the end of each move. In Scrabble, for example,
each player holds seven tiles at a time.

WordsWorth also allows you to play any tile from the pool without
drawing a small set of tiles each move. The pool can also be
unlimited, allowing you to play any letter you like at
any time, until both players get stuck anyway.

\subsection{Making a Move}

The basic idea behind making moves was described earlier. There 
are also some other types of moves. For example, if you cannot
make a good word, you can pass your turn. If you do so, you have
the option of returning any or all of your current tiles to the
pool, and drawing replacements (assuming you are drawing tiles,
and not playing freely from the pool as described in the previous 
section).

There is a special tile that plays a similar role to a joker in
cards. This tile is blank, and can be used to represent any
character. If you have such a tile and play it, you must state
what character it is representing. In WordsWorth, this tile is
represented by the letter it symbolises in lower case, or by the
underscore character `{\tt \_}' if it has no symbolic counterpart. For example, if you play the
word `{\tt STOP}' but use a blank tile to represent the `{\tt O}', then you would
enter this as `{\tt SToP}'. On the other hand, if you were passing your
turn and discarded a blank, you would enter it as `{\tt \_}', as it does
not represent any particular character (other than itself) when
being discarded. 

WordsWorth optionally allows either player to pick up a blank 
tile from the board if the letter it symbolises is placed in its
place by that player. Thus, if you played `{\tt SToP}', and the program
held an `{\tt O}' tile, it could exchange its `{\tt O}' for the blank. This
may constitute a full move on its own, a part of a pass move, or
simply something that can be done at the start of any move; you
specify whether you want these moves to be allowed at all, and 
if so, when.

The first move may optionally have some constraints placed on how
it should be played. For example, it could be specified that the
word must be played in the third row, or that it must cross the
center square, or that it may be played anywhere at all.

As mentioned earlier, usually each move other than the first move
must build on to the existing words like a crossword. Other
possibilities are territorial play (where each player builds on
their own words), and free-form play, in which play may occur
anywhere. In the former case, attachment to the opponents words
is allowed, but every play must include a word with a previously-
placed letter that was placed by the same player.

\section{Command Line Options}

Version 1.2 has the following command line syntax:

{\tt ww} [{\tt -D}] [{\tt -B0}$\mid${\tt -B1}] [{\tt -s}$\mid${\tt
-S}[$<${\em n}$>$]$\mid${\tt -Q}[$<n>$]] [{\tt -c}] [{\tt -f} $<${\em config
file}$>$] [$<${\em game file}$>$]

where:

\begin{description}
\item [{\tt -D}] shows you the computer's tiles (only on boards in which at least
one edge has length 15 or more).
\item [{\tt -B0}] makes the program play the first move, while {\tt -B1}
lets you play first. Otherwise the choice as to who starts is random.
\item [{-s}], {\tt -S} and {\tt -Q} are used for testing the performance 
of a particular configuration of the move selection algorithm. There are a
number of factors that affect how the program chooses its
move that can be modified within the configuration file. When
one of the command line arguments is given, the program plays
against itself, with the `player' taking your part being a
control which chooses its moves based on score only.
You can thus evaluate your algorithm configuration.

{\tt -S} and {\tt -s} show you the game as it is played. The {\tt -S} option
is faster, and exits automatically at the end of the game.
The {\tt -s} option waits for you to press a key between each move.

You can return to normal play from a {\tt -S} or {\tt -s} game by pressing
a key other than the {\tt ENTER} key. You can enter or resume autoplay
by selecting the `{\tt W} ' (watch) move option.

{\tt -Q} is the fastest and does not actually display the games,
just the end result. Both {\tt -S} and {\tt -Q} can take a numeric argument,
which is the number of games to play.
\item [{\tt -c}] colours tiles you place on the board in a different colour to
those placed by the program.
\item [{\tt -f}] allows you to load up a config file by name, rather than the default
`{\tt ww.cfg}' file.
\end{description}

The $<${\em game file}$>$ allows you to proceed straight into a saved game,
rather than starting a new game. See section~\ref{savedgame} for more details.


\section{Playing the Game}

When the game begins, WordsWorth uses a random number to decide who
plays first.If it is to play first, it will make a move quite quickly.
Either way, it will then be your turn to play. You can override the 
random choice of first player with the {\tt -B0} and {\tt -B1} command line options.

Your tiles will be shown at the bottom of the screen. If you
have a blank tile, it will be displayed as `{\tt \_}'.

When it is your move, you have numerous options, selected by pressing
the first letter of the option. We describe each in turn:

\subsection{C(onsult) - Consult the Dictionary}

This command lets you generate lists of anagrams and words, given
a set of letters. The set of letters you specify can include blank tiles,
as well as things like `any vowel' or `any non-vowel'.

There are a number of possibilities that you can specify. WordsWorth
will ask you whether you want the letters to be used in the order given, 
or whether you want anagrams of the letters as well - you specify
`{\tt A}' or
`{\tt O}'. Then you will be asked whether all the letters must be used (that is,
if you specified six letters only six-letter words would be listed),
or whether you are interested in words of any length in which case 
WordsWorth will use all combinations of choices out of those letters.
Press `{\tt A}' for `must play all tiles', or `{\tt S}' for `must play some tiles'.

You will then be asked for a pattern. WordsWorth will list all words 
that match the pattern (or match rearrangements of the pattern, in the
case of anagrams). Use `{\tt \_}' or `{\tt ?}' for a blank tile. You can also specify
a range of letters enclosed within square brackets, such as `{\tt
[A-Z]}' 
(equivalent to `{\tt \_}'), `{\tt [AEIOU]}' (any one vowel) or `{\tt
[A-MZ]}' (any one of A through M, or Z). To specify the opposite of 
a range, start the range with an exclamation mark. For example, 
{\tt [!AEIOU]} matches any non-vowel. If a particular letter (or 
letter range) {\em must} be in the listed words, precede it with a
`{\tt *}'.

Here are some examples:

\begin{description}
\item [{\tt C A A DIVIDE\_}] will find all seven letter words that can be made
with the tiles `{\tt DDEIIV}' and blank.
\item [{\tt C O A DIVIDE\_}] will find all seven letter words that start 
with the six letters `{\tt DIVIDE}'.
\item [{\tt C O A DIVIDE}] will just list `{\tt DIVIDE}'.
\item [{\tt C A A DIVIDE}] will list all anagrams of the word 
`divide'.
\item [{\tt C A S DIVIDE}] will list all words of up to seven letters
that can be made from the letters `{\tt DDEIIV}'.
\item [{\tt C A A AB\_}] will list all three-letter words containing 
an `{\tt A}' and `{\tt B}'.
\item [{\tt C O A ?[MN]?}] will list all three-letter words with an 
{\tt M} or {\tt N} in the center position.
\item [{\tt C A A [A-C][D-F]G}] will list all three-letter words containing
a {\tt G}, one of {\tt A},{\tt B}, or {\tt C}, and one of {\tt D},
{\tt E} or {\tt F}.

\item [{\tt C O A [A-C][D-F]G}] will list all three-letter words that begin
with an {\tt A}, {\tt B} or {\tt C}, have a {\tt D}, {\tt E} or {\tt
F} in the middle, and end with a {\tt G}.
\item [{\tt C A S ABJEIPN*[BHOS]}] will list all words that can be made out
of the seven letters `{\tt ABJEIPN}' plus one of {\tt B}, {\tt H},
{\tt O} or {\tt S}.
At least one {\tt B}, {\tt H}, {\tt O} or {\tt S} must be used.
\item [{\tt C A A ???*X}] will list all words of up to four letters that
contain an {\tt X}.
\item [{\tt C A S [!AEIOU][!AEIOU][!AEIOU][!AEIOU][!AEIOU]*[AEIOU]*[AEIOU]}]
will list all words of up to seven letters that contain exactly two vowels.
\end{description}

The pattern can be up to 79 characters in length.

WordsWorth displays up to 7 words at a time, and then waits for you 
to press a key, before displaying the next eight. If you want to 
return to the menu before the matching process is finished, press
the {\tt ESC} key at any point during matching.

If you request anagrams of a large number of letters, WordsWorth
may search for a while without producing any output. You must
either be patient, or press {\tt ESC} and do a simpler consultation.

\subsection{H(int)- Request a Hint}

WordsWorth will display a move you could make in this situation.
Unless WordsWorth recommends discarding tiles, you will be asked
if you want to play the suggested move. Press `y' or `Y' to do so;
press `h' or `H' to see another hint, or press any other key to take
you back to the list of choices.

You can modify the `ControlStrategy' configuration element to change
the way in which WordsWorth chooses hints.


\subsection{U(ndo) - Undo Moves}

WordsWorth will undo the last two moves made, if possible. As you
are being prompted for your move, this means the move the it just
made, and your move that preceded it.

\subsection{R(edo) - Replay Undone Moves}

WordsWorth will replay two undone moves, if there are such moves.
Note that if you undo all the way in a game in which WordsWorth plays
first, or you or WordsWorth play some new moves after a series of Undo's,
the Redo information is lost as it is no longer applicable to the state
of the game.

\subsection{Q(uit) - Quit and Return to DOS}

You will be asked for confirmation.

\subsection{S(ave) - Save the Current Sequence of Moves}

You will be asked for a file name to be used for the saved game. 

\subsection{L(oad) - Load a Saved Game}

You will be asked for the name of the saved game file to load.

\subsection{A(cross) - Play Across Move or D(own) - Play Down Move}

Both of these are similar, and are described together.

If you have already played an exchange this turn (see section~\ref{xchange}),
and the value `{\tt ExchangeAllowed}' in the configuration file is not 4, you will 
get a message saying ``You must pass!''.  

Otherwise, you will be asked for the row in which the first
letter of your word falls ({\em not} the first letter which you
are {\em playing}, just the first letter, which may already be on 
the board). You should enter a letter corresponding to the row
(there are labels on the left of the board for rows, and along the
top for columns). You will then be asked for the column. In each 
case, just press the key (the only time it is necessary to press
{\tt ENTER} is for multiple-letter input, such as words, tiles to discard,
and file names).

You will then be asked for the word. Type the word and press {\tt
ENTER}.
If your word contains blank tiles, you must type the letters they
represent, {\em in lower case}. All other letter should be typed in
{\em UPPER CASE}. For example, if you are playing the word `hello',
and you are using a blank for the `h', you enter `{\tt hELLO}'.

WordsWorth will then validate your move. If the word is not found in
the dictionary, you will be asked if you want it accepted anyway.
There are a number of other checks that will be performed. If any of
them fail, an error message will be displayed. The possible errors
are:

\begin{tabbing}
xxxx\=xxxxxxxx\=\+\kill
1st word must cross row $<${\em n}$>$\\
1st word must cross col $<${\em n}$>$\\
\mbox{}\\
\>You are playing the first word on the board and you have to play\\
\>it according to constraints which you have violated. These constraints\\
\>are specified in the configuration file.\\
\mbox{}\\
Word has no anchor square!\\
\mbox{}\\
\>The word does not attach to the existing word structure on the\\
\>board. You must build on to what is already there (but see below).\\
\mbox{}\\ 
Word conflicts with board!\\
\mbox{}\\
\>You entered a word which contains a letter which conflicts with\\
\>a letter already on the board.\\
\mbox{}\\
You don't have a blank!\\
\mbox{}\\
\>The word you entered requires that you have a blank tile, which\\
\>you don't.\\
\mbox{}\\
You don't have a $<${\em char}$>$\\
\mbox{}\\
\>The word you entered requires that you have a $<${\em char}$>$ \\
\>tile, which you don't.\\
\mbox{}\\	
Bad Xword\\
\mbox{}\\
\>The word you entered does not form all valid cross words.\\
\mbox{}\\
Can't start there!\\
\mbox{}\\
\>The word you entered must have its first letter preceded by an\\
\>empty square or the edge of the board, not another letter.\\
\mbox{}\\
Can't end there!\\
\mbox{}\\
\>The word you entered must have its last letter followed by an\\
\>empty square or the edge of the board, not another letter.\\
\end{tabbing}

If one of these errors occurs, you will simply be asked to enter
your move again. You will be given the option of overriding
WordsWorth's dictionary lookups of the main word and any cross words,
in case you wish to play a move that is not in the dictionary.

{\em If you make mistakes}:

\begin{verse}
\begin{itemize}
\item in entering the word, you can use the backspace
key to make corrections before pressing {\tt ENTER}
\item press the {\tt ESC} key once to clear the current line
and twice to cancel the current operation and return to the menu.
\end{itemize}
\end{verse}

Note that every word played must include at least one anchor
square, but the definition of what constitutes an anchor
square varies. The default is that it is any empty square
with at least one non-empty neighbour. If you select free
form play, all empty squares are anchors (except possibly on the first move). If you select territorial
play, the usual definition applies except that only neighbouring
tiles played by the same player are examined.


\subsection{P(ass) - Pass Turn, and (optionally) Discard some Tiles}

You will be asked which tiles to discard. Enter the list in any 
order. All letters must be entered in upper case, and if you 
discard a blank (although its a rare occasion) you must enter 
`{\tt \_}'. WordsWorth will validate your input. If you have all the
tiles specified, they will be returned to the pool and a new set
will be drawn. Otherwise you will be asked to enter your move again.

\subsection{X(change) - Pick up Blank Tile from Board}
\label{xchange}

This option lets you pick up a blank from the board if you hold the
letter it represents, which you use to replace it. There are a number
of options, which are selected by the {\tt ExchangeAllowed} configuration
variable. The possible selections are:

\begin{verbatim}
ExchangeAllowed = 0     # No exchange moves are allowed (in which case
                        # the X option does not appear on the menu)
ExchangeAlowed = 1      # an exchange constitutes an entire move
ExchangeAlowed = 2      # an exchange can be followed by a tile exchange
                        # with the pool, but not with a word play
ExchangeAlowed = 3      # an exchange can be done before any move.
\end{verbatim}

Exchanges don't have any score associated with them.

\subsection{F1 - Show Help}

Pressing {\tt F1} at any time will display some help
about what input WordsWorth is currently expecting
from you.

\subsection{N(ext move) - Make Program Play a Different Move}

As with hints, the program stores up to twenty moves when it
decides what to play. If you don't like the move it made (maybe it
took a triple word score?!) you can force it to play differently.
Unlike hints, which are shown to you in order of score, the
program's moves are ordered by weight.

\subsection{W(atch) - Start or Resume Autoplay Mode}

Selecting this option tells the program to start playing for you.
You can interrupt autoplay by pressing any key other than {\tt ENTER}.

\subsection{V(iew) - Toggle Opponent Tile View On/Off}

Pressing `{\tt V}' toggles the display of the opponent's tiles on and
off.

\subsection{E(xclude) - Disallow Word from Play}
\label{exclude}

If you press `{\tt E}' you will be asked for a word to exclude
from play. The default is the last word played by the computer.
The word will be added to a list of up to 16 words within WordsWorth,
as well as being appended to a file called {\tt WORDS.EX}. If you
run the `{\tt REBUILD.BAT}' batch file, it will convert this {\tt
WORDS.EX}
list into a {\tt WORDS.NO} dictionary. This dictionary has the capacity
for about 100-200 words, depending on length.

To summarise - you use the `{\tt E}' command to exclude a word that isn't
already in your `{\tt WORDS.NO}' dictionary. After quitting WordsWorth,
if you have used the `{\tt E}' command, you should run `{\tt REBUILD}' to 
update the `{\tt WORDS.NO}' file. You can also add new words yourself
to the `{\tt WORDS.EX}' file, and then run `{\tt REBUILD}'.

After using this command, WW will do an undo, as it assumes that
the `{\tt E}' was due to its last move.

\section{Saved Games}
\label{savedgame}

WordsWorth allows games to be saved and loaded. Games are saved
in plain text form, with comments, so its quite easy to see what is
going on.

It is possible to have a board which has letters placed on it before
you even begin playing. One application of this feature would
be to set up a partially completed crossword puzzle and get WordsWorth
to find possible completions.

When you load a game, it will be played through to the point at
which it was saved, after which you can continue. The undo command
can also be used to go back to an earlier point.

The format of saved games is described in the Appendix.	Note that 
it is up to you to ensure that the game you load is compatible with
the current configuration. The results otherwise will be unpredictable
at best.

\section{Ending the Game}

When both players are stuck, or the tiles are all finished, WordsWorth
exits and prints out the end score. You can also exit by 
entering `{\tt q}' at the move prompt. WordsWorth will clean up the
screen and print out a summary of the games results at the time
of exit.

\section{Known Problems with this Release}

\begin{itemize}
\item  When undoing moves, the owner attribute is not undone. That is,
if an exchange move is done in which one player picks up
a blank played by the other, and this move is then undone.
the program continues to regard the original tile as being
played by the player who picked it up, not the original player.
This affects the display when each player's tiles are
displayed in a different colour, and territorial play.
\item The game record/load/save/undo/redo logic and data structures
are overly complex and should be redesigned at some stage.
\end{itemize}

\begin{center}
\bf
ENJOY PLAYING WORDSWORTH!
\end{center}
\end{document}
