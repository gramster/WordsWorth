
Please find enclosed a demo copy of WordsWorth, a package I have
written in an attempt to carve a career in the shareware software
market (not an easy task!) I hope that you will try the package
out, and find it of sufficient interest to review in your
publication.

WordsWorth is a package consisting of two programs, which together
constitute a flexible package for crossword fans and Scrabble players.
The first program is essentially a Scrabble game, although it has
been made much more flexible than Scrabble, in that the user can 
modify the score rules, the board layout, the number of tiles, and 
so on. The full version of the program allows the user to save and
load games in progress, undo and redo any number of moves, get
hints, consult the dictionary with a powerful pattern-lookup
facility, force the program to play a different move, and so on.
Some of these features are disabled in the demo version, which
also has a much smaller dictionary (17,000 words) than the 
registered version (82,000 words). The program features attractive
graphics and supports Hercules, EGA, VGA and SuperVGA video adapters.
The program is driven from the keyboard rather than with the mouse
as this is more efficient.

The second program is a stand-alone version of the dictionary
consultation facility of the game, but includes three additional
features: multi-word anagram generation (the game consult only
does single-word anagrams), block word generation (forming a block 
in which every row and every column is a valid word), and crossword
solving. This last feature is especially interesting. The user can
enter the layout of a crossword board (the layout editor includes
flip/rotate actions to speed up this process), and mark the start of
new words of multiple word clues. Then, as the user solves a clue,
he or she enters the word into the grid. While this is happening,
the program continually computes a set of constraints on the board,
based on the dictionary. For example, the program may determine that
a certain cross word can only one of a certain subset of words, 
each of which has either an O or an A in the second position. This
fact constrains the set of words which can go down through that
second position, which in turn constrains other words, and so on.
The result of computing these constraints is the set of letters
that may be entered in each empty square of the crossword. As the
user moves around the board, the program displays the allowed letters
for the current square as well as the possible down and across words
that can pass through the square. Once about a third of the crossword
is complete in a particular area of the grid, the program makes it 
easy to find the remaining words, and also fills in any squares that
are constrained to a single character automatically. In some case
the program has correctly completed crosswords that were only
half entered.

The two programs are sold together for R50, via mail order 
from myself at the above address.

Thanks for your time. I hope you find WordsWorth as much fun
as I have (especially putting the names of your friends and enemies
through the multi-word anagram generator! Just a couple of gems:
VEGETARIAN = EATING RAVE, TELEVISION = EVIL ION SET).

